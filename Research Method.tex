\documentclass[conference]{IEEEtran}
\IEEEoverridecommandlockouts
% The preceding line is only needed to identify funding in the first footnote. If that is unneeded, please comment it out.
%Template version as of 6/27/2024

\usepackage{cite}
\usepackage{multirow}
\usepackage{array} 
\usepackage{amsmath,amssymb,amsfonts}
\usepackage{algorithmic}
\usepackage{graphicx}
\usepackage{textcomp}
\usepackage{xcolor}
\def\BibTeX{{\rm B\kern-.05em{\sc i\kern-.025em b}\kern-.08em
    T\kern-.1667em\lower.7ex\hbox{E}\kern-.125emX}}
\begin{document}

\title{The Impact of Academic, Work, and Financial Stress on Depression Among University Students\\}

\author{\IEEEauthorblockN{1\textsuperscript{st} Ibrahim Jelii Wale}
\IEEEauthorblockA{\textit{University of Lincoln} \\
\textit{Lincolnshire}\\
United kingdom \\
brymorla@gmail.com}
}

\maketitle

\begin{abstract}
This report examines the substantial impact of academic, work, and financial stress on the depression rate among university students. Current research indicates a strong correlation between each of these stressors and the development of depressive symptoms. The report further explores how a family history of mental illness may increase a student's susceptibility to stress and depression related to academic life. It also evaluates the effectiveness of various mental health treatments and support systems available to students. 

The analysis employed logistic regression and comparative analysis (T-test) on a sample of $N = 27,900$. The relationships between depression and academic pressure ($\mathrm{OR} = 2.30$, $p < 0.001$), as well as financial stress ($\mathrm{OR} = 1.78$, $p < 0.001$), were statistically significant. In contrast, work pressure was not a significant predictor ($p = 0.203$). A family history of mental illness modestly enhanced the observed effects. 

These findings underscore the urgent need for targeted mental health interventions, particularly those addressing academic and financial stressors, in order to support students who are most at risk.
\end{abstract}

\begin{IEEEkeywords}
depression, academic stress, financial stress, moderation analysis, mental health
\end{IEEEkeywords}

\section{Introduction}

The mental health of university students remains a serious global challenge, and mental disorders—especially depression—are becoming increasingly common~\cite{b1}. This rise in incidence over the last decade signals a pressing need for universities to investigate the underlying causes and implement support mechanisms~\cite{b2}. Depression is particularly widespread among students and has serious consequences, including poor academic performance, lower retention rates, and reduced quality of life~\cite{b3}. Addressing the wide range of stressors that contribute to depression is critical to promoting a healthy academic environment.

University students often face a combination of academic, work-related, and financial pressures that uniquely impact their mental well-being~\cite{b4}. The university setting marked by demanding academics, transitions into adulthood, financial independence, and future career anxiety amplifies these stressors~\cite{b5}. For many, university life involves adjusting to new social networks, coping with a more rigorous academic schedule, managing personal finances for the first time, and confronting career-related uncertainty. The convergence of these developmental challenges creates an ideal setting for the emergence of mental health problems.

This paper conducts an in-depth analysis of how academic, work-related, and financial stress contribute to depression among university students. It also examines the moderating role of family history of mental illness, evaluates the effectiveness of existing mental health interventions, identifies trends in depression prevalence within the student population, and discusses implications for university policies and student support services.

\subsection{Impact of Academic Stress and Depression}

Multiple studies have consistently demonstrated a strong positive correlation between academic stress and depression in university students~\cite{b6}. Across diverse demographic groups, academic stress emerges as a major precursor to depressive symptoms. One study reported a particularly strong relationship between academic workload and depression ($r = 0.845$), emphasizing that increased academic pressure significantly correlates with depressive outcomes~\cite{b4}. These findings are consistent with earlier research showing associations between academic stress, anxiety, and depression, and have been validated by national studies of representative samples of U.S. college students~\cite{b6}.

\subsection{Impact of Work-Related Stress on Students’ Mental Health}

A growing number of students are balancing work alongside academic responsibilities, creating considerable mental health challenges~\cite{b5}. The traditional image of a full-time student focused solely on academics is becoming outdated; many now work out of economic necessity to meet basic needs and reduce long-term debt from student loans~\cite{b6}. A significant proportion of both full-time and part-time students are employed during their studies~\cite{b5}. Higher education institutions must recognize these challenges and expand mental health support frameworks beyond traditional academic-focused services to better support working students.

\subsection{Financial Strain as a Predictor of Depression}

Financial stress is a prominent and well-documented predictor of depression in university populations~\cite{b6}. The cost of higher education including tuition, textbooks, housing, and daily living expenses frequently drives students into debt, fostering chronic stress~\cite{b7}. Many students report constant anxiety over affording these essentials, reflecting the widespread nature of financial pressure. This persistent financial strain places students at increased risk for mental health issues, underscoring the need for institutional awareness and targeted support programs.



\subsection{Research Hypotheses}
\textbf{H\textsubscript{1}}: Work pressure, academic pressure, and financial pressure are positively associated with depression.\\

\textbf{H\textsubscript{2}}: These associations are stronger among students with a family history of mental illness.

\section{Methods}

\subsection{Data Collection}
This study utilized a secondary analysis of anonymized survey data concerning university students. The dataset, which is publicly archived and compliant with the General Data Protection Regulation (GDPR), was sourced from the Kaggle platform. Prior to analysis, data cleaning and preparation were conducted using Microsoft Excel 2019. Additional metadata and documentation about the dataset are available in an associated GitHub repository.

\subsection{Variables}
\textbf{Outcome:} Depression (binary: self-reported diagnosis or clinical threshold).\\
\textbf{Predictors:} Academic pressure, work pressure, financial stress.\\
\textbf{Mean difference of family history:} Family history of mental illness (binary).

\subsection{Statistical Analysis}

A binary logistic regression analysis was conducted to assess the influence of academic pressure, financial stress, and work pressure on the likelihood of depression among university students. Depression variable served as the dependent variable with binary outcomes (``yes'' or ``no''). The overall model showed strong predictive ability, correctly classifying 76.0\% of all cases, with a specificity of 82.3\% and sensitivity of 67.2\%. This indicates that the model was more effective at identifying students who were not depressed.

Academic pressure emerged as the most significant predictor of depression, with an odds ratio (OR) of 2.30 (95\% CI [2.25, 2.36], $p < 0.001$). This implies that each one-unit increase in academic stress more than doubled the odds of experiencing depressive symptoms. Even when controlling for other types of stress in the model, academic pressure remained the most influential factor in predicting student mental health outcomes.

Financial stress was identified as a secondary but still significant factor in the onset of depression. For each unit increase in financial stress, the odds of depression increased by approximately 78\% (OR = 1.78, 95\% CI [1.74, 1.82], $p < 0.001$). This result reflects the substantial emotional burden that financial instability can place on students, particularly in relation to tuition, living costs, and limited financial resources.

Work pressure, however, did not significantly predict depression status. Although the odds ratio was 1.43, the associated $p$-value was not significant ($p = 0.203$), and the 95\% confidence interval [0.83, 2.47] included the null value. The regression coefficient ($B = 0.357$) was positive, yet statistically insignificant, suggesting that employment-related stress did not substantially contribute to depressive symptoms in this sample.

The classification table confirmed the model's stronger ability to identify non-depressed students. Of the 11,565 students classified as depressed, the model correctly predicted 67.2\%, while it accurately identified 82.3\% of the 16,335 students who were not depressed. This disparity suggests that while the model is proficient at detecting low-risk individuals, it may benefit from refinement to better capture high-risk cases.

The logistic regression findings reinforce earlier t-test results, confirming that academic and financial stress are key contributors to depression among university students. While work pressure showed no significant impact, the two identified stressors (academic and financial) were both strongly associated with increased depressive symptoms. These results highlight the critical importance of implementing targeted interventions in academic and financial support to improve student mental health outcomes.

\section{Results}

\subsection{Primary Analysis (H\textsubscript{1})}
The logistic regression model revealed the following associations between the independent stressors and the likelihood of depression: \\
Table I and Table II revealed by the logistic regression show the distribution of deression and the classifcation of depression predition with the overal prediction accuracy.

\begin{itemize}
    \item \textbf{Academic Pressure:} Table III revealed that academic pressure was the most significant predictor of depression, with an odds ratio of $OR = 2.30$, 95\% CI [2.25, 2.36], $p < 0.001$.
    \item \textbf{Financial Stress:} After academic pressue, financial stress happens to be the next significant predictor, with $OR = 1.78$, 95\% CI [1.74, 1.82], $p < 0.001$.
    \item \textbf{Work Pressure:} Tablle III also revealed that work pressure variable was not a statistically significant predictor of depression, $OR = 1.43$, $p = 0.203$.
\end{itemize}

The logistic model demonstrated good predictive performance, correctly classifying 76\% of cases, with a specificity of 82.3\% and a sensitivity of 67.2\%.


\begin{table}[htbp]
\caption{Depression Frequency Distribution}
\centering
\begin{tabular}{|p{1.5cm}|r|r|r|r|}
\hline
\textbf{Depression} & \textbf{Freq.} & \textbf{Percent\%} & \textbf{Valid \%} & \textbf{Cumul. \%} \\
\hline
Yes & 11,565 & 41.5 & 41.5 & 41.5 \\
No  & 16,335 & 58.5 & 58.5 & 100.0 \\
\hline
\textbf{Total} & 27,900 & 100.0 & 100.0 & -- \\
\hline
\end{tabular}
\label{tab:depression-frequency}
\end{table}


\begin{table}[htbp]
\caption{Classification Table for Depression Prediction}
\centering
\begin{tabular}{|p{2cm}|p{1.1cm}|p{1.1cm}|p{1.4cm}|}
\hline
\textbf{Observed} & \textbf{Predicted Yes} & \textbf{Predicted No} & \textbf{\% Correct} \\
\hline
Depression Yes & 7,776 & 3,787 & 67.2\% \\
Depression No  & 2,897 & 13,437 & 82.3\% \\
\hline
\multicolumn{3}{|l|}{\textbf{Overall Accuracy}} & \textbf{76.0\%} \\
\hline
\end{tabular}
\label{tab:classification}
\end{table}


\begin{table*}[htbp]
\caption{Logistic Regression Results: Variables in the Equation}
\centering
\begin{tabular}{|l|r|r|r|r|r|r|r|r|}
\hline
\textbf{Variable} & \textbf{B} & \textbf{S.E.} & \textbf{Wald} & \textbf{df} & \textbf{Sig.} & \textbf{Exp(B)} & \textbf{95\% CI Lower} & \textbf{Upper} \\
\hline
Academic Pressure & 0.835 & 0.012 & 4769.017 & 1 & 0.000 & 2.304 & 2.250 & 2.359 \\
Financial Stress  & 0.574 & 0.011 & 2805.313 & 1 & 0.000 & 1.775 & 1.738 & 1.814 \\
Work Pressure     & 0.357 & 0.280 & 1.623    & 1 & 0.203 & 1.428 & 0.825 & 2.472 \\
Constant          & -3.940 & 0.055 & 5181.430 & 1 & 0.000 & 0.019 & --    & --    \\
\hline
\end{tabular}
\label{tab:variables-in-equation}
\end{table*}

\vspace{1mm}
\noindent\textit{Note:} B = unstandardized coefficient; S.E. = standard error; Wald = Wald chi-square test statistic; df = degrees of freedom; Sig. = significance level; Exp(B) = odds ratio; CI = confidence interval.


\subsection{T-test by Family History of Mental Illness (H\textsubscript{2})}

To evaluate whether the relationship between depression and specific stress domains namely academic pressure, work pressure, and financial stress varies according to family history of mental illness, independent-samples t-tests were conducted for students with and without such a history. The findings consistently demonstrated statistically significant differences in academic and financial stress between depressed and non-depressed students across both groups, while work-related stress did not show significant variation.

Among students \textbf{without} a family history of mental illness, depressed individuals reported markedly lower levels of academic pressure ($M = 2.36$, $SD = 1.26$) than their non-depressed counterparts ($M = 3.68$, $SD = 1.20$), with a highly significant difference ($p < 0.001$) and a narrow 95\% confidence interval (CI) for the mean difference [$-1.36$, $-1.28$]. A similar pattern was observed for financial stress, where depressed students ($M = 2.55$, $SD = 1.34$) scored significantly lower than non-depressed peers ($M = 3.58$, $SD = 1.33$), $p < 0.001$, 95\% CI [$-1.08$, $-0.99$]. In contrast, there was no significant difference in reported work pressure between depressed ($M = 0.0008$) and non-depressed ($M = 0.0006$) students ($p = 0.864$), suggesting that work-related stress may be a less relevant factor for depression in this group.

Among students \textbf{with} a family history of mental illness, the same pattern emerged. Depressed students experienced significantly lower academic pressure ($M = 2.36$, $SD = 1.25$) than those without depression ($M = 3.71$, $SD = 1.18$), with a highly significant $p$-value ($< 0.001$) and a tight CI [$-1.39$, $-1.31$]. Financial stress also significantly differentiated depressed ($M = 2.48$, $SD = 1.36$) and non-depressed ($M = 3.58$, $SD = 1.33$) students, $p < 0.001$, CI [$-1.14$, $-1.05$]. Work pressure, again, was not a significant differentiator ($p = 0.209$), even though the difference in means was slightly larger than in the no-history group (0.0004 vs. 0.0000).

Importantly, the magnitude of the mean differences in both academic and financial stress was consistent across the two groups, with academic stress differences averaging approximately $-1.3$ and financial stress differences around $-1.1$. A direct comparison of these mean differences across family history groups yielded a non-significant $p$-value of 0.992, suggesting no meaningful interaction effect between family history and the relationship of stress to depression. This indicates that while a family history of mental illness may increase an individual's general vulnerability to depression, it does not significantly alter how academic or financial stressors contribute to depressive symptoms.

These findings have several implications. First, they highlight the universal nature of academic and financial stress as major risk factors for depression, regardless of familial predisposition. Second, they suggest that while genetic or familial vulnerability may be a background risk factor, the situational and environmental pressures associated with university life play a more direct and measurable role in shaping students’ mental health outcomes. Finally, the non-significance of work stress across both groups may reflect the relatively lower prevalence or intensity of employment obligations among university students compared to academic and financial demands or it may suggest that students are better equipped to manage work responsibilities, or perceive them as less central to their identity and future prospects.

In sum, the results strongly support the hypothesis that academic and financial stress are critical determinants of depressive symptoms among university students, and that their impact is largely independent of family history of mental illness. Interventions targeting these stress domains may therefore be broadly effective across diverse student populations, regardless of individual psychiatric backgrounds.


\begin{table*}[htbp]
\caption{Comparison of Stress Factors by Depression Status (Mental Illness History = No)}
\begin{center}
\begin{tabular}{|l|c|c|c|c|c|c|c|}
\hline
\textbf{Factor} & \textbf{Depression} & \textbf{N} & \textbf{Mean} & \textbf{Std. Dev.} & \textbf{Std. Error} & \textbf{Sig. (2-tailed)} & \textbf{95\% CI (Lower, Upper)} \\
\hline
\multirow{2}{*}{Academic Pressure} & Yes & 6335 & 2.3640 & 1.25515 & 0.01577 & \multirow{2}{*}{0.000} & -1.35637, -1.27582 \\
                                   & No  & 8062 & 3.6801 & 1.19856 & 0.01335 &                          &                      \\
\hline
\multirow{2}{*}{Work Pressure}     & Yes & 6335 & 0.0008 & 0.06282 & 0.00079 & \multirow{2}{*}{0.864} & -0.00177, 0.002109 \\
                                   & No  & 8062 & 0.0006 & 0.05569 & 0.00062 &                         &                     \\
\hline
\multirow{2}{*}{Financial Stress}  & Yes & 6335 & 2.5478 & 1.33799 & 0.01681 & \multirow{2}{*}{0.000} & -1.08, -0.9921 \\
                                   & No  & 8061 & 3.5838 & 1.33351 & 0.01485 &                        &                  \\
\hline
\end{tabular}
\label{tab:depression_factors}
\end{center}
\vspace{-1em}
\begin{flushleft}
\footnotesize Depression status significantly affected Academic Pressure and Financial Stress ($p < .001$), but not Work Pressure ($p = .864$). Equal variances were assumed across groups.
\end{flushleft}
\end{table*}


\begin{table*}[htbp]
\caption{Comparison of Stress Factors by Depression Status (Mental Illness History = Yes)}
\begin{center}
\begin{tabular}{|l|c|c|c|c|c|c|c|}
\hline
\textbf{Factor} & \textbf{Depression} & \textbf{N} & \textbf{Mean} & \textbf{Std. Dev.} & \textbf{Std. Error} & \textbf{Sig. (2-tailed)} & \textbf{95\% CI (Lower, Upper)} \\
\hline
\multirow{2}{*}{Academic Pressure} & Yes & 5230 & 2.3587 & 1.25036 & 0.01729 & \multirow{2}{*}{0.000} & -1.38902, -1.30541 \\
                                   & No  & 8273 & 3.7059 & 1.17926 & 0.01297 &                        &                     \\
\hline
\multirow{2}{*}{Work Pressure}     & Yes & 5230 & 0.0004 & 0.02766 & 0.00038 & \multirow{2}{*}{0.209} & -0.00021, 0.00098 \\
                                   & No  & 8273 & 0.0000 & 0.00000 & 0.00000 &                        &                    \\
\hline
\multirow{2}{*}{Financial Stress}  & Yes & 5228 & 2.4836 & 1.35702 & 0.01877 & \multirow{2}{*}{0.000} & -1.13819, -1.04520 \\
                                   & No  & 8273 & 3.5752 & 1.33323 & 0.01466 &                        &                   \\
\hline
\end{tabular}
\label{tab:depression_factors_mih_yes}
\end{center}
\vspace{-1em}
\begin{flushleft}
\footnotesize  Among participants with a history of mental illness, depression was associated with significantly higher Academic Pressure and Financial Stress ($p < .001$), while Work Pressure showed no significant difference ($p = .209$). Overall there is no significant difference between students with/without mental history P-value = 0.992
\end{flushleft}
\end{table*}

\section{Discussion}
\subsection{Key Findings}
\begin{itemize}
    \item Academic and financial stress are key predictors of depression, as predicted by Lazarus \& Folkman's (1984) stress-coping theory.
    \item Family history plays a marginal aggravating role, suggesting genetic/environmental vulnerabilities.
\end{itemize}

\subsection{Implications}
\textbf{Policy:} Academic/financial support should be prioritized in universities (e.g., flexible deadlines, emergency grants).\\
\textbf{Research:} Replicate with longitudinal designs to permit causality inferences.

\subsection{Limitations}
\begin{itemize}
    \item Cross-sectional data limits causal inference.
    \item Self-reported stressors may be prone to bias.
\end{itemize}

\section{Conclusion}
This study confirms that academic and financial stress are major correlates of depression among university students. In contrast, work pressure did not exhibit a significant relationship with depression levels. Furthermore, there were no substantial differences in depression outcomes between students with or without a family history of mental illness, leading to the rejection of the null hypothesis \text{H\textsubscript{2}}. 

These findings highlight the importance of implementing targeted mental health interventions aimed specifically at alleviating academic and financial stressors. Such efforts may help mitigate the risk of depression and prevent mental health crises within student populations.

\begin{thebibliography}{00}

\bibitem{b1} N. Liangruenrom, M. Joshanloo, W. Hutaphat, and S. Kittisuksathit, “Prevalence and correlates of depression among Thai university students: nationwide study,” BJPsych Open, vol. 11, no. 2, p. e59, Mar. 2025, doi: 10.1192/bjo.2025.21.
\bibitem{b2}“New brief on student mental health support in higher education released | International Institute for Higher Education in Latin America and the Caribbean.” Accessed: Apr. 23, 2025. [Online]. Available: https://www.iesalc.unesco.org/en/articles/new-brief-student-mental-health-support-higher-education-released
\bibitem{b3} R. P. Auerbach et al., “WHO world mental health surveys international college student project: Prevalence and distribution of mental disorders,” J Abnorm Psychol, vol. 127, no. 7, pp. 623–638, Oct. 2018, doi: 10.1037/abn0000362.
\bibitem{b4} “The Consequences of Academic Pressure on Students’ Mental Health.” Accessed: Apr. 23, 2025. [Online]. Available: https://www.transformationsnetwork.com/post/the-consequences-of-academic-pressure-on-students-mental-health
\bibitem{b5} “Hidden Cost of Working Through College: Students Struggle to Balance Mental Health – The Red Line Project.” Accessed: Apr. 23, 2025. [Online]. Available: https://redlineproject.news/2024/11/30/hidden-cost-of-working-through-college-students-struggle-to-balance-mental-health/
\bibitem{b6} “Understanding The Impact Of Financial Stress In College Students | BetterHelp.” Accessed: Apr. 23, 2025. [Online]. Available: https://www.betterhelp.com/advice/stress/understanding-the-impact-of-financial-stress-in-college-students/
\bibitem{b7} “Depression in College Students | EBSCO Research Starters.” Accessed: Apr. 25, 2025. [Online]. Available: https://www.ebsco.com/research-starters/consumer-health/depression-college-students
 
\end{thebibliography}
\vspace{12pt} 

\textbf{Ethics Statement}: This study used anonymized data; no ethical approval was required. \\
\textbf{Conflicts of Interest}: None declared.

\end{document}
